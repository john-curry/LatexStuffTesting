\nonstopmode
\documentclass[12pt,jou]{apa}
\usepackage{times}
\usepackage[margin=1.0in]{geometry}
\usepackage{apacite}
\bibliographystyle{apacite}
\usepackage{setspace}
\usepackage{fancyhdr}
\doublespacing
\setlength{\parindent}{1cm}
\begin{document}
\begin{titlepage}
\begin{center}
\textsc{\huge Psychopathy and it's Potential Benefits}\\[1.5cm]

\begin{minipage}{0.4\textwidth}
\begin{flushleft} \large
John \textsc{Curry}\\
Camosun Collage
\end{flushleft}
\end{minipage}
\vfill
{\large \today}
\end{center}
\end{titlepage}


\pagenumbering{gobble}
\pagenumbering{arabic}
\pagestyle{fancy}
\fancyhf{}
\fancyhead[R]{\thepage}
\fancyhead[L]{Psychopathy Benefits}
\section{Abstract}
\textit{
Discussed is some of the basic research and understanding of
psychopathy and how it is classified (ie the PCL-R). Then we will discuss the
finer points of the classification process. The modern understanding tends to be
more towards the psychopaths that find them themselves in prison. There are
alternatives to treatment of these individuals. A lack of empathy is discussed
in a useful context. Further benefits are described for the trait of willingness
to manipulate. 
}
\section{Psychopathy Benefits} 
Psychopaths are individuals who have a certain set of personality traits.
 The specific personality traits that these individuals tend
to exhibit include deception, irresponsibility, lack of forward planning,
impulsivity, lack of empathy, lack of guilt, antisocial behaviour, and
stimulation seeking behaviour ~\cite{brazil}. It is not hard to see the effects
a psychopath could have in any number of situation. When diagnosing
psychopathic individuals, the most used and widely respected measure for a
persons psychopathy is the \textit{Have Psychopathy Checklist - Revised
}(PCL-R) \cite{nickerson2014}. The PCL-R measures a wide range of personality
traits and ranks them on a scale of 0 to 2, adds them up, and gives the person
a score out of 40. Anyone over a score of 30 is labeled a psychopath. The
specific traits are as follows "glib and superficial charm, exaggerated
grandiosity, need for stimulation, pathological lying, cunning and
manipulativeness, lack of remorse or guilt; shallow affect (superficial
emotional responsiveness), callousness and lack of empathy, parasitic
lifestyle, poor behavioral controls, sexual promiscuity, early behavior
problems, lack of realistic long-term goals, impulsivity, failure to accept
responsibility for own actions, many short-term marital relationships, juvenile
delinquency, revocation of conditional release, and criminal versatility”
\cite{hareharpur1991}. Don't be surprised if you can think of a few people one
can fit a fair amount of theses criteria; the psychopath is a large sum of all
these traits and not just a subset of them. These traits tend to lead an
individual with them to a life of criminality and imprisonment. 

The key point to get for the PCL-R is that a psychopathy is a range of values.
A person can still rank high on the scales if they lack a couple of the less
useful traits. Something like a lack of unrealistic goals would be very
detrimental to anyone, but they would still rank high on the PCL-R. Another
trait that would be not so beneficial in most situations would be poor
behavioural control, but the person would still classify as a psychopath. A
combination of theses traits, would still affect a large subset of the
psychopath community, but would be less prone to criminal, or
anti-social behaviour. 

The research done on psychopaths tends to be done of incarcerated individuals
who have managed to get themselves prison time because of their ailment. This
has led to a bias in the research data. Because psychopaths have a tendency to
get themselves into prison, research has been focused to the psychopaths in
prison and diagnostic tools used to detect psychopaths were developed to detect
psychopathy in prisoners. The research is slightly biased towards individuals
who have a criminal history. Although a psychopath is more likely to go to
prison then the non-psychopathic individual, there is a large about of
psychopaths, labeled Successful Psychopaths, that have managed to successfully
navigate away from the criminal justice system. These individuals will be the
focus of this paper. 

Psychopaths are usually considered hard, if not impossible, to treat
\cite{crimpsych}. As an alternative we should be looking for ways to integrate
theses individuals into society. This paper will argue that due to a psychopath's
lack of empathy, and willingness to manipulate, they can be
productive members of society.

A psychopaths lack of empathy is arguably one of their most obvious traits,
and one that could easily lead to a psychopath ending up on the wrong side of
the criminal justice system. A psychopath doesn't have the ability to think
about another person and how their actions affect others. People who have
empathy may take a problem, and consider how it affects their coworkers, friends,
family, and then get to solving the problem. This may lead to a complex solution
to what may be a simple problem, because due to the complications of human
relations, may end up become a problem that is infinitely harder to deal with. A
psychopath, on the other hand, will look at the problem in terms of mostly
logic, and will not be held back by the emotional baggage of the people related
to the problem. % TODO: find a reference for how psychopaths use their logic
                % part of their brain to process emotional information

Willingness to manipulate can be a useful trait in many situations. Pesky
environmental regulations getting in the way of building your multi-million dollar
shopping mall? A psychopath will have no problem lying to the government and
telling them that the building meets all environmental standards. Is the UN
calling you out for bombing your neighbouring terrorist state? Well just send
your friendly neighbourhood psychopath to go tell the UN that it has been the
enemy guerrilla fighters that are bombing civilian targets. Just like a lack of
empathy can get past certain social boundaries, so to can a willingness to lie
and deceive get you past the need for telling the truth. No longer will you have
to publicly admit your wrongdoings when you have someone to lie for you. This
leads to having a buffer between you and the law. If anyone ever catches the
psychopath lying, you can deny any involvement and place the blame solely on the
psychopath. 

These are just some short examples where psychopaths can be productive members
of society. These traits, not found in the general population, can be useful to
anyone who can realize their potential. The cost of doing business in the modern
era is sometimes fraught with needless bureaucracy, social responsibility and
other peoples feelings. Why not have someone who can get past these impasses for
you. When your business if getting a little dirty, why not have someone who
doesn't mind getting a little blood on their hands. Besides, what is more
important then making money? Nothing. 
\newpage
\bibliography{philass}
\end{document}
